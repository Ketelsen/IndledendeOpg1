\documentclass[12pt,a4paper]{article}
\usepackage[utf8]{inputenc}
\usepackage[danish]{babel}
\usepackage{amsmath}
\usepackage{amsfonts}
\usepackage{amssymb}
\usepackage{graphicx}
\author{Anders Christian Ketelsen, s133195 Gustav Valdemar Fjorder, s153651 Philip August Kisling s164421}
\title{Boblesortering}
\begin{document}
\maketitle
\begin{abstract}
Dette dokument omhandler boblesortering. Der beskrives algoritmen og præsenteres en kompleksitetsanalyse.
\end{abstract}
\section{Introduktion}
Boblesortering (\textsl{eng. bubble sort}) er en populær sorteringsalgoritme og er en af de simpleste algoritmer at forstå og implementere. Dog er den ikke en særlig effektiv sorteringsalgoritme\footnote{Mere om dette i ``Algoritmer og Dataskruturer 1''}; hverken for store eller små lister, og den anvendes sjældent i praksis. Boblesortering sorterer, som navnet antyder, elementerne i en liste ved at \textsl{boble} hvert element gennem listen til sin rette plads i listen.

\section{Analyse af boblesortering}
Antallet af sammenligninger, som boblesortering udfører på en tabel af længde $n$, er i værste fald
\[\sum_{i=1}^{n-1}i=1+2+3+...+n-1=\frac{n(n-1)}{2}\]
I bedste fald er antallet $n-1$. Se tabel ~\ref{Antal sammenligninger for boblesortering.}.

\begin{figure}
\centering
\includegraphics[scale=0.5]{Nyhavn1}\caption{Illustration af boblesortering.}\label{fig:nyhavn}
\end{figure}

\begin{tabular}{|r|l|}
\hline
\textbf{Værst} & $n(n-1)/2$\\
\hline
\textbf{Bedst}& $n-1$\\

\end{tabular}

\end{document}